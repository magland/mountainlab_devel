\documentclass{article}
\usepackage{amsmath}
\usepackage{amsfonts}

\begin{document}

The centroid of a collection of points $x_1,\dots,x_n\in \mathbb{R}^M$ is given by
$$C(x)=\frac{1}{n}\sum_{i=1}^n x_i.$$

In our context the $x_i$ form a cluster of spike events and $C(x)$ is a representative waveform. The problem with the centroid is that it is not robust with respect to outliers.

An alternative is the geometric median. This is given by
$$G(x)=\arg \min_y \sum_{i=1}^n\|x_i-y\|,$$
where the sum is over the Euclidean distances to the data points. The geometric median is robust against outliers and even behaves well in the case where the data comprise two clusters. As shown in the example below, the geometric median will generally lie toward the center of the primary cluster, whereas the centroid is heavily influenced by outliers.

There is no known formula for computing the geometric mean. However, there is a straightforward iterative algorithm called the Weisfeld algorithm, which is simply iteratively re-weighted least squares. The idea is that minimizing the sum of the squared distances (rather than the distances themselves) is easy. In fact, it is just the centroid! Weighted least-squares is also easy. So the idea is to iteratively choose the weights to be equal to the recipricols of the distances:

$$y_{n+1}=\frac{\sum_i \|x_i-y_n\|^{-1}x_i}{\sum_i \|x_i-y_n\|^{-1}}.$$

I think that this sequence always converges to the geometric median, as long as we don't end up dividing by zero. Who wants to prove it?

\end{document}